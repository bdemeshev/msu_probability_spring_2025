% arara: xelatex
\documentclass[12pt]{article}

% \usepackage{physics}

\usepackage{hyperref}
\hypersetup{
    colorlinks=true,
    linkcolor=blue,
    filecolor=magenta,      
    urlcolor=cyan,
    pdftitle={Overleaf Example},
    pdfpagemode=FullScreen,
    }

\usepackage{tikzducks}

\usepackage{tikz} % картинки в tikz
\usepackage{microtype} % свешивание пунктуации

\usepackage{array} % для столбцов фиксированной ширины

\usepackage{indentfirst} % отступ в первом параграфе

\usepackage{sectsty} % для центрирования названий частей
\allsectionsfont{\centering}

\usepackage{amsmath, amsfonts, amssymb} % куча стандартных математических плюшек

\usepackage{mathtools}
\usepackage{comment}

\usepackage[top=2cm, left=1.2cm, right=1.2cm, bottom=2cm]{geometry} % размер текста на странице

\usepackage{lastpage} % чтобы узнать номер последней страницы

\usepackage{enumitem} % дополнительные плюшки для списков
%  например \begin{enumerate}[resume] позволяет продолжить нумерацию в новом списке
\usepackage{caption}

\usepackage{url} % to use \url{link to web}


\newcommand{\smallduck}{\begin{tikzpicture}[scale=0.3]
    \duck[
        cape=black,
        hat=black,
        mask=black
    ]
    \end{tikzpicture}}

\usepackage{fancyhdr} % весёлые колонтитулы
\pagestyle{fancy}
\lhead{}
\chead{}
\rhead{Счастье 2025}
\lfoot{Можно использовать чит-лист А4 и простой калькулятор, продолжительность: 120 минут. }
\cfoot{}
\rfoot{}

\renewcommand{\headrulewidth}{0.4pt}
\renewcommand{\footrulewidth}{0.4pt}

\usepackage{tcolorbox} % рамочки!

\usepackage{todonotes} % для вставки в документ заметок о том, что осталось сделать
% \todo{Здесь надо коэффициенты исправить}
% \missingfigure{Здесь будет Последний день Помпеи}
% \listoftodos - печатает все поставленные \todo'шки


% более красивые таблицы
\usepackage{booktabs}
% заповеди из докупентации:
% 1. Не используйте вертикальные линни
% 2. Не используйте двойные линии
% 3. Единицы измерения - в шапку таблицы
% 4. Не сокращайте .1 вместо 0.1
% 5. Повторяющееся значение повторяйте, а не говорите "то же"


\setcounter{MaxMatrixCols}{20}
% by crazy default pmatrix supports only 10 cols :)


\usepackage{fontspec}
\usepackage{libertine}
\usepackage{polyglossia}

\setmainlanguage{russian}
\setotherlanguages{english}

% download "Linux Libertine" fonts:
% http://www.linuxlibertine.org/index.php?id=91&L=1
% \setmainfont{Linux Libertine O} % or Helvetica, Arial, Cambria
% why do we need \newfontfamily:
% http://tex.stackexchange.com/questions/91507/
% \newfontfamily{\cyrillicfonttt}{Linux Libertine O}

\AddEnumerateCounter{\asbuk}{\russian@alph}{щ} % для списков с русскими буквами
\setlist[enumerate, 2]{label=\asbuk*),ref=\asbuk*}

%% эконометрические сокращения
\DeclareMathOperator{\Cov}{\mathbb{C}ov}
\DeclareMathOperator{\BestLin}{BestLin}
\DeclareMathOperator{\Corr}{\mathbb{C}orr}
\DeclareMathOperator{\Var}{\mathbb{V}ar}
\DeclareMathOperator{\pCorr}{\mathrm{pCorr}}
\DeclareMathOperator{\col}{col}
\DeclareMathOperator{\row}{row}
\DeclareMathOperator{\plim}{plim}
\DeclareMathOperator{\pe}{pe}



\let\P\relax
\DeclareMathOperator{\P}{\mathbb{P}}

\DeclarePairedDelimiter{\abs}{\lvert}{\rvert}

\let\H\relax
\DeclareMathOperator{\H}{\mathbb{H}}


\DeclareMathOperator{\E}{\mathbb{E}}
% \DeclareMathOperator{\tr}{trace}
\DeclareMathOperator{\card}{card}

\DeclareMathOperator{\Convex}{Convex}

\newcommand \cN{\mathcal{N}}
\newcommand \dN{\mathcal{N}}
\newcommand \dBin{\mathrm{Bin}}
\newcommand \dUnif{\mathrm{Unif}}


\newcommand \RR{\mathbb{R}}
\newcommand \NN{\mathbb{N}}

\newcommand{\dBern}{\mathrm{Bern}}



\begin{document}



\begin{enumerate}
\item У меня в ящике три кубика: два — правильных, а на третьем — три двойки и три четвёрки. 
Я извлекаю из ящика равновероятно два кубика, а затем подбрасываю каждый из них по одному разу. 
Обозначим результаты бросков как $X_1$ и $X_2$.
\begin{enumerate}
    \item {[3]} Найдите $\P(X_2 = 2 \mid X_1 = 2)$. 
    \item {[3]} Правда ли, что величины $X_1$ и $X_2$ независимы?
    \item {[4]} Какова вероятность того, что оба выбранных кубика — правильные, если оба раза выпала двойка?
\end{enumerate}


\item Джеймс Бонд десантируется в случайную точку $(X, Y)$ равномерно выбираемую на периметре треугольника 
с вершинами $(0, 0)$, $(0, 1)$ и $(4, 0)$.

\begin{enumerate}
    \item {[5]} Найдите $\E(Y \mid X = 0)$ и $\E(Y \mid X > 0)$.
    \item {[5]} Найдите $\E(Y \mid X)$ и $\Var(Y \mid X)$.
\end{enumerate}

\item Величины $(X_n)$ независимы и экспоненциально распределены с интенсивностью $\lambda$.

\begin{enumerate}
  \item {[5]} Найдите предел 
  $\plim (X_1^2 + X_2^2 + \dots + X_n^2)/(3n + 100)$.
  \item {[5]} Найдите предел 
  $\plim \sum_{i=1}^n(X_i - \bar X)^2 / n, \text{ где } \bar X = (X_1 + X_2 + \dots + X_n) / n$.
\end{enumerate}

\item  Сейчас акция стоит 100 рублей. 
Каждый день цена может равновероятно и независимо от предыстории либо возрасти на 8\%, либо упасть на 5\%.
\begin{enumerate}
\item {[5]} Какова вероятность того, что через 64 дня цена будет больше 110 рублей?
\item {[5]} Какой порог цены акции через 64 дня будет превышен с вероятностью $0.1$?
\end{enumerate}

Уточнение: $\ln(1.08) = 0.077$, $\ln(0.95) = -0.051$, $\ln(1.1) = 0.095$, запишите ответы с использованием функции распределения и посчитайте их по таблице.

\item Величины $X_1$ и $X_2$ независимы, их плотности, соответственно, равны $f_1(x)$ и $f_2(x)$.
\begin{enumerate}
    \item {[3]} Найдите совместную функцию плотности вектора $(X_1, X_2)$.
    Запишите соответствующий элемент вероятности $\pe(x_1, x_2)$. 
    \item {[4]} Найдите совместную функцию плотности вектора $(X_1, S)$, где $S = X_1 + X_2$.
    Запишите соответсвующий элемент вероятности $\pe(x_1, s)$.
    \item {[3]} Запишите функцию плотности величины $S$ в виде интеграла. 
\end{enumerate}

Уточнение: под элементом вероятности мы подразумеваем дифференциальную форму $\pe(x, y) = f(x, y) dx \wedge dy$.

\item Кржмелик загадал два различных натуральных числа, написал их на бумажках и взял одну бумажку в левую руку, а другую — в правую.
Вахмурка выбирает одну руку Кржемелика равновероятно наугад и узнает число, написанное на этой бумажке.
Задача Вахмурки — отгадать, большее или меньшее из двух чисел он узнал.

У Вахмурки есть правильная монетка. 
\begin{enumerate}
    \item {[2]} Предложите Вахмурке стратегию, которая гарантирует вероятность угадывания ровно $0.5$ вне зависимости от действий Кржмелика.
    \item {[8]} Предложите Вахмурке стратегию, которая гарантирует вероятность угадывания строго больше $0.5$ вне зависимости от действий Кржмелика.
\end{enumerate}
\end{enumerate}

\end{document}

