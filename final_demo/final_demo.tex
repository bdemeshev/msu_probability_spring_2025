% arara: xelatex
\documentclass[12pt]{article}

% \usepackage{physics}

\usepackage{hyperref}
\hypersetup{
    colorlinks=true,
    linkcolor=blue,
    filecolor=magenta,      
    urlcolor=cyan,
    pdftitle={Overleaf Example},
    pdfpagemode=FullScreen,
    }

\usepackage{tikzducks}

\usepackage{tikz} % картинки в tikz
\usepackage{microtype} % свешивание пунктуации

\usepackage{array} % для столбцов фиксированной ширины

\usepackage{indentfirst} % отступ в первом параграфе

\usepackage{sectsty} % для центрирования названий частей
\allsectionsfont{\centering}

\usepackage{amsmath, amsfonts, amssymb} % куча стандартных математических плюшек

\usepackage{mathtools}
\usepackage{comment}

\usepackage[top=2cm, left=1.2cm, right=1.2cm, bottom=2cm]{geometry} % размер текста на странице

\usepackage{lastpage} % чтобы узнать номер последней страницы

\usepackage{enumitem} % дополнительные плюшки для списков
%  например \begin{enumerate}[resume] позволяет продолжить нумерацию в новом списке
\usepackage{caption}

\usepackage{url} % to use \url{link to web}


\newcommand{\smallduck}{\begin{tikzpicture}[scale=0.3]
    \duck[
        cape=black,
        hat=black,
        mask=black
    ]
    \end{tikzpicture}}

\usepackage{fancyhdr} % весёлые колонтитулы
\pagestyle{fancy}
\lhead{}
\chead{}
\rhead{Демо версия, экзамен 2025}
\lfoot{}
\cfoot{}
\rfoot{}

\renewcommand{\headrulewidth}{0.4pt}
\renewcommand{\footrulewidth}{0.4pt}

\usepackage{tcolorbox} % рамочки!

\usepackage{todonotes} % для вставки в документ заметок о том, что осталось сделать
% \todo{Здесь надо коэффициенты исправить}
% \missingfigure{Здесь будет Последний день Помпеи}
% \listoftodos - печатает все поставленные \todo'шки


% более красивые таблицы
\usepackage{booktabs}
% заповеди из докупентации:
% 1. Не используйте вертикальные линни
% 2. Не используйте двойные линии
% 3. Единицы измерения - в шапку таблицы
% 4. Не сокращайте .1 вместо 0.1
% 5. Повторяющееся значение повторяйте, а не говорите "то же"


\setcounter{MaxMatrixCols}{20}
% by crazy default pmatrix supports only 10 cols :)


\usepackage{fontspec}
\usepackage{libertine}
\usepackage{polyglossia}

\setmainlanguage{russian}
\setotherlanguages{english}

% download "Linux Libertine" fonts:
% http://www.linuxlibertine.org/index.php?id=91&L=1
% \setmainfont{Linux Libertine O} % or Helvetica, Arial, Cambria
% why do we need \newfontfamily:
% http://tex.stackexchange.com/questions/91507/
% \newfontfamily{\cyrillicfonttt}{Linux Libertine O}

\AddEnumerateCounter{\asbuk}{\russian@alph}{щ} % для списков с русскими буквами
\setlist[enumerate, 2]{label=\asbuk*),ref=\asbuk*}

%% эконометрические сокращения
\DeclareMathOperator{\Cov}{\mathbb{C}ov}
\DeclareMathOperator{\BestLin}{BestLin}
\DeclareMathOperator{\Corr}{\mathbb{C}orr}
\DeclareMathOperator{\Var}{\mathbb{V}ar}
\DeclareMathOperator{\pCorr}{\mathrm{pCorr}}
\DeclareMathOperator{\col}{col}
\DeclareMathOperator{\row}{row}
\DeclareMathOperator{\plim}{plim}


\let\P\relax
\DeclareMathOperator{\P}{\mathbb{P}}

\DeclarePairedDelimiter{\abs}{\lvert}{\rvert}

\let\H\relax
\DeclareMathOperator{\H}{\mathbb{H}}


\DeclareMathOperator{\E}{\mathbb{E}}
% \DeclareMathOperator{\tr}{trace}
\DeclareMathOperator{\card}{card}

\DeclareMathOperator{\Convex}{Convex}

\newcommand \cN{\mathcal{N}}
\newcommand \dN{\mathcal{N}}
\newcommand \dBin{\mathrm{Bin}}
\newcommand \dUnif{\mathrm{Unif}}


\newcommand \RR{\mathbb{R}}
\newcommand \NN{\mathbb{N}}

\newcommand{\dBern}{\mathrm{Bern}}



\begin{document}

\section*{Формат}

В экзамене будет 6 задач:
\begin{enumerate}
    \item Условная вероятность.
    \item Условное математическое ожидание. 
    \item Сходимость по вероятности и закон больших чисел.
    \item Центральная предельная теорема. 
    \item Сюрприз!
    \item Сюрприз!
\end{enumerate}
Задачи имеют равный вес. 
Продолжительность работы 120 минут. 
Можно использовать чит-лист А4 и простой калькулятор.

\section*{Демо «Сцилла»}
\begin{enumerate}
\item У меня два кубика: один правильный, а на втором — две двойки и четыре четвёрки. 
Сначала я равновероятно выбираю один из кубиков, а потом подкидываю его два раза. 
Обозначим результаты бросков как $X_1$ и $X_2$.
\begin{enumerate}
    \item Найдите $\P(X_2 = 2 \mid X_1 = 2)$. 
    \item Правда ли, что величины $X_1$ и $X_2$ независимы?
    \item Какова вероятность того, что был выбран правильный кубик, если оба раза выпала двойка?
\end{enumerate}


\item Джеймс Бонд десантируется в случайную точку $(X, Y)$ с совместной плотностью 
\[
f(x, y) = \begin{cases}
    cx + y, \text{ если } x\in [0, 1], y\in [0, 1], \\
    0, \text{ иначе.}
\end{cases}
\]
\begin{enumerate}
    \item Найдите константу $c$. 
    \item Найдите $\E(Y \mid X = 1)$, $\E(Y^2 \mid X= 1)$, $\Var(Y \mid X = 1)$.
    \item Найдите $\E(Y \mid X)$.
\end{enumerate}

\item Величины $(X_n)$ независимы и нормально распределены $\cN(5; 10)$.

\begin{enumerate}
  \item Найдите предел по вероятности
  \[
  \plim \frac{X_1^2 + X_2^2 + \dots + X_n^2}{7n}.
  \]
  \item Найдите предел по вероятности
  \[
  \plim \ln(X_1^2 + X_2^2 + \dots + X_n^2) - \ln n.
  \]
\end{enumerate}

\item Каждый день цена акции равновероятно поднимается или опускается на один рубль. 
Сейчас акция стоит 1000 рублей.
\begin{enumerate}
    \item Найдите вероятность того, что через год акция будет стоить больше 1030 рублей.
    \item Найдите такой порог цены $h$, выше которого цена акции через год окажется с вероятностью $0.3$.
\end{enumerate}

Уточнение: в году 365 дней, в пунктах (а) и (б) запишите ответ тремя способами.
Во-первых, с помощью определённого интеграла, во-вторых, в виде кода на любом языке программирования, 
в-третьих, получите ответ с помощью таблицы нормального распределения. 

\item 
\item 



\end{enumerate}

\section*{Демо «Харибда»}
\begin{enumerate}
\item Илон Маск подбрасывает правильную монетку четыре раза. 
\begin{enumerate}
    \item Правда ли, что число орлов в первых трёх бросках и число орлов в последних трёх бросках независимы?
    \item Какова вероятность того, что все четыре раза выпал орёл, если орёл выпал хотя бы два раза?
    \item Какова вероятность того, что все четыре раза выпал орёл, если орёл выпал хотя бы два раза в первых трёх бросках и хотя бы два раза в последних трёх бросках?
\end{enumerate}

\item Колобок стартует в точке $(X_0=0, Y_0=0)$.
За каждую минуту он откатывается на единицу вверх с вероятностью $0.2$, вниз — с вероятностью $0.2$, вправо — с вероятностью $0.3$, влево — с вероятностью $0.3$.
Все перекатывания независимы. 
Обозначим координаты Колобка через $t$ минут как $(X_t, Y_t)$.
\begin{enumerate}
    \item Найдите $\E(Y_2 \mid X_2 = 1)$, $\E(Y_2^2 \mid X_2 = 1)$, $\Var(Y_2 \mid X_2 = 1)$.
    \item Найдите $\E(Y_2 \mid X_2)$.
\end{enumerate}

\item Величины $(X_n)$ независимы и равномерно распределены $\dUnif[0, 1]$.

\begin{enumerate}
  \item Найдите предел по вероятности
  \[
  \plim \frac{nX_1 + X_2 + X_3  + X_4 + \dots + X_n}{7n}.
  \]
  \item Найдите предел по вероятности
  \[
  \plim \sqrt[2n]{X_1 X_2 X_3 \dots X_n}.
  \]
\end{enumerate}

\item  Истинная вероятность выпадения монетки «орлом» равна $0.63$.
\begin{enumerate}
\item Какова вероятность, что в 100 испытаниях выборочная доля выпадения орлов будет
отличаться от истинной вероятности менее, чем на $0.07$?
\item Каким должно быть минимальное количество испытаний, чтобы вероятность отличия
выборочной доли и истинной вероятности менее чем на $0.02$ была больше $0.95$?
\end{enumerate}

\item 
\item 



\end{enumerate}

\end{document}

