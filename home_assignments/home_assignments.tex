% arara: xelatex
\documentclass[12pt]{article}

% Econometrics, 2024-2025

% \usepackage{physics}

\usepackage{verse}

\usepackage{tikzducks}

\usepackage{tikz} % картинки в tikz
\usepackage{microtype} % свешивание пунктуации

\usepackage{array} % для столбцов фиксированной ширины

\usepackage{indentfirst} % отступ в первом параграфе

\usepackage{sectsty} % для центрирования названий частей
\allsectionsfont{\centering}

\usepackage{amsmath, amsfonts, amssymb} % куча стандартных математических плюшек

\usepackage{comment}

\usepackage[top=2cm, left=1.2cm, right=1.2cm, bottom=2cm]{geometry} % размер текста на странице

\usepackage{lastpage} % чтобы узнать номер последней страницы

\usepackage{enumitem} % дополнительные плюшки для списков
%  например \begin{enumerate}[resume] позволяет продолжить нумерацию в новом списке
\usepackage{caption}

\usepackage{hyperref} % loads url
% \usepackage{url} % to use \url{link to web}
\urlstyle{same}


\newcommand{\smallduck}{\begin{tikzpicture}[scale=0.3]
    \duck[
        cape=black,
        hat=black,
        mask=black
    ]
    \end{tikzpicture}}

\usepackage{fancyhdr} % весёлые колонтитулы
\pagestyle{fancy}
\lhead{}
\chead{}
\rhead{Домашние задания для самураев}
\lfoot{}
\cfoot{}
\rfoot{}

\renewcommand{\headrulewidth}{0.4pt}
\renewcommand{\footrulewidth}{0.4pt}

\usepackage{tcolorbox} % рамочки!

\usepackage{todonotes} % для вставки в документ заметок о том, что осталось сделать
% \todo{Здесь надо коэффициенты исправить}
% \missingfigure{Здесь будет Последний день Помпеи}
% \listoftodos - печатает все поставленные \todo'шки


% более красивые таблицы
\usepackage{booktabs}
% заповеди из докупентации:
% 1. Не используйте вертикальные линни
% 2. Не используйте двойные линии
% 3. Единицы измерения - в шапку таблицы
% 4. Не сокращайте .1 вместо 0.1
% 5. Повторяющееся значение повторяйте, а не говорите "то же"


\setcounter{MaxMatrixCols}{20}
% by crazy default pmatrix supports only 10 cols :)


\usepackage{fontspec}
\usepackage{libertine}
\usepackage{polyglossia}

\setmainlanguage{russian}
\setotherlanguages{english}

% download "Linux Libertine" fonts:
% http://www.linuxlibertine.org/index.php?id=91&L=1
% \setmainfont{Linux Libertine O} % or Helvetica, Arial, Cambria
% why do we need \newfontfamily:
% http://tex.stackexchange.com/questions/91507/
% \newfontfamily{\cyrillicfonttt}{Linux Libertine O}

\AddEnumerateCounter{\asbuk}{\russian@alph}{щ} % для списков с русскими буквами
% \setlist[enumerate, 2]{label=\asbuk*),ref=\asbuk*}

%% эконометрические сокращения
\DeclareMathOperator{\Cov}{\mathbb{C}ov}
\DeclareMathOperator{\Corr}{\mathbb{C}orr}
\DeclareMathOperator{\Var}{\mathbb{V}ar}
\DeclareMathOperator{\hVar}{\widehat{\mathbb{V}ar}}
\DeclareMathOperator{\col}{col}
\DeclareMathOperator{\row}{row}

\let\P\relax
\DeclareMathOperator{\P}{\mathbb{P}}

\DeclareMathOperator{\E}{\mathbb{E}}
% \DeclareMathOperator{\tr}{trace}
\DeclareMathOperator{\card}{card}

\DeclareMathOperator{\Convex}{Convex}

\newcommand \cN{\mathcal{N}}
\newcommand \RR{\mathbb{R}}
\newcommand \NN{\mathbb{N}}

\newcommand{\dU}{\mathrm{Unif}}
\newcommand{\dUnif}{\mathrm{Unif}}

\DeclareMathOperator{\loss}{loss}

\newcommand{\hy}{\hat y}
\newcommand{\hb}{\hat\beta}

\newcommand{\addtag}[1]{}

\usepackage{mathtools}
\DeclarePairedDelimiter{\norm}{\lVert}{\rVert}
\DeclarePairedDelimiter{\abs}{\lvert}{\rvert}
\DeclarePairedDelimiter{\scalp}{\langle}{\rangle}


\begin{document}

\section*{Домашнее задание 1}

Дедлайн: 2025-03-20, 21:00.

Оцениваемая часть:

\begin{enumerate}
\item В красном мешке у Деда Мороза 5 красных и 4 синих шара, а в синем мешке — 3 красных и 10 синих шаров. 
Сначала Дед Мороз выбирает один из мешков равновероятно. 
Затем Дед Мороз достаёт из выбранного мешка один шар.
А затем Дед Мороз достаёт ещё два шара из \emph{другого} мешка.

Обозначим $R$ — общее число красных извлечённых шаров, и $B$ — общее число синих шаров.

\begin{enumerate}
    \item Составьте табличку распределения случайной величины $R$.
    \item Найдите $\P(R \text{ — чётное})$, $\E(R)$, $\E(2B + 7)$, $\E(R\cdot B)$.
    \item Найдите $\P(R \geq 1, B \geq 1)$, $\E(R \cdot I(B\geq 1))$.
\end{enumerate}

Напоминалочка: $I(A)$ — индикатор события $A$, случайная величина, равная $1$, если событие $A$ произошло и $0$ — иначе.


\item У Илона Маска две монетки: $A$-монетка выпадает орлом с вероятностью $0.3$, $B$-монетка выпадает орлом с вероятностью $0.4$.
Каждая из монеток выпадает либо решкой, либо орлом. 
Всего Илон делает $100$ подбрасываний. 
Сначала Илон Маск подбрасывает монетку $A$. 
Затем он действует по простому правилу: если выпал орёл, то следующей будет подброшена монетка $A$, если выпала решка,
то следующией будет подброшена монетка $B$.
Обозначим $X$ — общее число выпавших орлов, $Y$ — общее число орлов выпавших на монетке $B$.

\begin{enumerate}
    \item Найдите $\E(X)$ и $\E(Y)$.
    \item Найдите $\E(XY)$.
\end{enumerate}

\end{enumerate}

Прекрасная неоцениваемая часть в удовольствие:

\begin{enumerate}[resume]
    \item У Маши две монетки: золотая и серебряная.
Сначала Маша подкидывает золотую монетку.
Если золотая монетка выпала орлом, то Маша подкидывает серебряную монетку один раз.
Если золотая монетка выпала решкой — то подкидывает серебряную два раза.

Пусть $X$ — общее количество выпавших орлов на золотой и серебряной монетках.

\begin{enumerate}
\item Найдите все возможные значения $X$ и их вероятности.
\item Каково ожидаемое количество выпавших орлов?
\end{enumerate}

    \item Вспомним свойство аддитивности вероятности. 
    $A$: Если задан набор несовместных событий $A_1$, $A_2$, \dots, ($A_i \cap A_j = \emptyset$ при $i\neq j$),
    то $\P(\cup A_i) = \sum_i \P(A_i)$.

    Докажите, что свойство аддитивности эквивалентно свойству $B$ и свойству $C$.

    $B$: Если задан набор вложенных событий $B_1 \subseteq B_2 \subseteq B_3 \dots$, то $\lim_i \P(B_i) = \P(\lim_i B_i)$.

    $C$: Если задан набор вложенных событий $\dots C_3 \subseteq C_2 \subseteq C_1$, то $\lim_i \P(C_i) = \P(\lim_i C_i)$.

    \item В шкатулке у Маши 100 пар серёжек. 
    Каждый день утром она выбирает одну пару наугад, носит ее, а вечером возвращает в шкатулку. 
    Проходит год.
    \begin{enumerate}
    \item Сколько в среднем пар окажутся ни разу не надетыми?
    \item Сколько в среднем пар окажутся надетыми не менее двух раз?
\end{enumerate}
    \item  Над озером взлетело 20 уток. 
    Каждый из 10 охотников один раз стреляет в случайно выбираемую им утку. 
    Охотники целятся одновременно, поэтому несколько охотников могут выбрать одну и ту же утку.
    Величина $Y$ — количество выживших уток, $X$ — количество попавших в цель охотников. 
   
   \begin{enumerate}
     \item Найдите $\E(X)$, $\E(Y)$, если охотники стреляют без промаха. 
     \item Как изменятся ответы, если вероятность попадания равна $0.7$?
   \end{enumerate}
   
\end{enumerate}
    

\section*{Домашнее задание 2}

Дедлайн: 2025-03-27, 21:00.

Оцениваемая часть:

\begin{enumerate}
\item Аллея из десяти каштанов скоро вся зацветёт!
Завтра каждый каштан может либо цвести, либо — нет, независимо от других.
Вероятность того, что $k$-й по счёту каштан цветёт равна $1/k$. 

\begin{enumerate}
    \item Найдите ожидаемое количество цветущих каштанов. 
    \item Если два каштана, стоящих рядом, цветут, то проходящий аллею Хосе де Рибас улыбается и говорит «Très bien!»
    Сколько раз Хосе в среднем скажет «Très bien»?     
\end{enumerate}

\item Мы подбрасываем правильную монетку до тех пор, пока впервые не выпадет последовательность $HHT$ или $THT$.
\begin{enumerate}
    \item Сколько бросков в среднем потребуется?
    \item Какова вероятность того, что эксперимент окончится последовательностью $HHT$?
    \item Сколько в среднем выпадет решек?
\end{enumerate}

\end{enumerate}

Прекрасная неоцениваемая часть в удовольствие:

\begin{enumerate}[resume]
    \item  У Пети есть монетка, выпадающая орлом с вероятностью $ p\in (0;1) $.
    У Васи есть монетка, выпадающая орлом с вероятностью $1/2$.
    Они одновременно подбрасывают свои монетки до тех пор,
    пока у них не окажется набранным одинаковое количество орлов.
    В частности, они останавливаются после первого подбрасывания,
    если оно дало одинаковые результаты. 
    
    Сколько в среднем раз им придётся подбросить монетку?
    \item Илье Муромцу\addtag{герои!Илья Муромец} предстоит дорога к камню. От камня начинаются ещё три дороги.
    Каждая из тех дорог снова оканчивается камнем. И от каждого камня начинаются ещё три дороги.
    И каждые те три дороги оканчиваются камнем\ldots И так далее до бесконечности.
    На каждой дороге живёт трёхголовый Змей Горыныч\addtag{герои!Змей Горыныч}.
    Каждый Змей Горыныч бодрствует независимо от других с вероятностью (хм, Вы не поверите!) одна третья.
    У Василисы Премудрой\addtag{герои!Василиса Премудрая} существует Чудо-Карта, на которой видно,
    какие Змеи Горынычи бодрствуют\addtag{герои!Змей Горыныч}, а какие — нет.
    
    \begin{enumerate}
    \item Како вероятность того, что Илья Муромец будет исключительно мимо спящих Змеев Горынычей, если каждый раз будет выбирать случайную дорогу на развилке?
    \item Какова вероятность того,
    что Василиса Премудрая\addtag{герои!Василиса Премудрая}  \emph{сможет найти на карте}
    бесконечный жизненный путь Ильи Муромца проходящий исключительно мимо спящих Змеев Горынычей?
    \end{enumerate}
    
    \item В каждой вершине треугольника по ёжику. Каждую минуту с вероятностью $0.5$ каждый ежик
    независимо от других двигается по часовой стрелке, с вероятностью
    $0.5$ — против часовой стрелки.
    Обозначим $T$ — время до встречи всех ежей в одной вершине.
    
    \begin{enumerate}
      \item Найдите $\P(T=1)$, $\P(T=2)$, $\P(T=3)$, $\E(T)$.
      \item Как изменятся ответы, если вероятность движения по часовой стрелке равна $p$?
    \end{enumerate}

   \item  Маша и Даша играют в следующую игру. 
 Правильный кубик подкидывают неограниченное число раз.
 Если на кубике\addtag{кубик} выпадает 1, 2 или 3, то соответствующее количество монет добавляется на кон.
 Если выпадает 4 или 5, то игра оканчивается и Маша получает сумму, лежащую на кону.
 Если выпадает 6, то игра оканчивается и Даша получает сумму, лежащую на кону. 
 Изначально на кону лежит ноль рублей.
\begin{enumerate}
\item Какова вероятность того, что игра рано или поздно закончится выпадением 6-ки?
\item Какова ожидаемая продолжительность игры?
\item Чему равен ожидаемый выигрыш Маши и ожидаемый выигрыш Даши?
\item Чему равны ожидаемые расходы организаторов игры?
\item Чему равен ожидаемый выигрыш Маши, если изначально на кону лежит 100 рублей?
\item Изменим изначальное условие: если выпадает 5, то сумма на кону сгорает, а игра продолжается.
Чему будет равен средний выигрыш Маши и средний выигрыш Даши в новую игру?
\end{enumerate}

   
\end{enumerate}


\section*{Домашнее задание 3}

Дедлайн: 2025-04-05, 21:00.

Оцениваемая часть:

\begin{enumerate}
\item Таблица совместного распределения пары величин $X$ и $Y$ имеет вид

\begin{tabular}{*{4}{c}}
\toprule
& $X=-1$ & $X=0$ & $X=1$ \\
\midrule
$Y=0$ & 0.1 & 0.2 & 0.3  \\
$Y=1$ & 0.2 & 0.1 & 0.1  \\
\bottomrule
\end{tabular}

\begin{enumerate}
    \item Найдите $\P(Y = 1 \mid X \geq 0)$, $\E(Y \mid X \geq 0)$.
    \item Найдите $\P(X \geq 0 \mid Y)$, $\E(Y \mid X)$,  $\E(X \mid Y)$.
    \item Найдите $\E(1 / (X + 2) \mid Y = 0)$, $\E(1 / (X + 2) \mid Y)$.
\end{enumerate}

\item Дональд Трамп подкидывает монетку бесконечное число раз. 
Монетка выпадает орлом с вероятность $0.4$ и решкой — с вероятностью $0.6$.
Обозначим $X$ — номер броска, когда впервые выпал орёл, а $Y$ — индикатор того, что орёл был в третьем броске.
\begin{enumerate}
    \item Найдите $\E(Y)$ и $\E(X)$.
    \item Найдите $\P(X \geq 5 \mid Y = 1)$ и $\P(X \geq 5 \mid Y)$.
    \item Найдите $\E(Y \mid X = 5)$, $\E(X \mid Y = 1)$.
    \item Найдите $\E(Y \mid X)$, $\E(X \mid Y)$.
\end{enumerate}

\end{enumerate}

Прекрасная неоцениваемая часть в удовольствие:

\begin{enumerate}[resume]
\item  Ген карих глаз доминирует ген синих. 
Следовательно, у носителя пары bb глаза синие, а у носителя пар BB и Bb — карие. 
У диплоидных организмов (а мы такие :)) одна аллель наследуется от папы, а одна — от мамы. 
В семье у кареглазых родителей два сына — кареглазый и синеглазый. 
Кареглазый женился на синеглазой девушке. 

Какова вероятность рождения у них синеглазого ребенка?

\item У Ивана Грозного $n$ бояр. 
Каждый боярин берёт мзду независимо от других с вероятностью $1/2$.

\begin{enumerate}
  \item Какова вероятность того, что все бояре берут мзду, если случайно выбранный боярин берёт мзду?
  \item Какова вероятность того, что все бояре берут мзду, если хотя бы один из бояр берёт мзду?
\end{enumerate}

\item На праздник ровно 5\% жён итальянских мафиози получили в подарок цветы.
Цветы получают в подарок только от мужа.
Также известно, что 0.5\% жён получили в подарок пирожные, причём половина жён получила их от мужа,
а половина — от брата.
Среди жён, получивших пирожные от мужа, 90\% получили в подарок цветы.
Среди жён, получивших пирожные от брата, 5\% получили в подарок цветы.

\begin{enumerate}
  \item Кармела получила в подарок цветы. Какова условная вероятность того, что она получила
      пирожные в подарок от мужа?

 \item Талия получила в подарок цветы и пирожные. Какова условная вероятность того, что она
получила пирожные в подарок от мужа?
\end{enumerate}

\item Задача Эльханана Мосселя.

Ты подбрасываешь кубик до первой шестерки.

\begin{enumerate}
 \item Чему равно ожидаемое общее количество сделанных за игру бросков?
  \item Чему равно ожидаемое общее количество сделанных за игру бросков, если за время игры ни разу не выпало нечётное число?
  \item Как изменится ответ, если за время игры было $a$ нечётных бросков?
\end{enumerate}


\end{enumerate}


\section*{Домашнее задание 4}

Дедлайн: 2025-04-19, 21:00.

Оцениваемая часть:

\begin{enumerate}
    \item Завтрашняя цена акции — случайная величина с функцией плотности  $f(x)=\frac{3}{4}
    \max \left\{x (2-x), 0 \right\}$.
    \begin{enumerate}
    \item Постройте график функции плотности;
    \item Обозначим Васино благосостояние завтра величиной $Y$. 
    Найдите $\E(Y)$, $\Var(Y)$ и функцию распределения $Y$ в каждой из ситуаций:
    \begin{enumerate}
    \item[A:] У Васи есть 10 акций;
    \item[B:] У Васи нет акций, но есть один опцион-пут на завтра со страйком $1.2$ рубля. 
    \end{enumerate}
    \end{enumerate}
    Опцион-пут — это право продать одну акцию по страйк-цене. 
    Если страйк-цена опциона-пут ниже фактической цены акции, то опцион-пут бесполезен, нет смысла пользоваться правом. 
    Однако, если страйк-цена опциона-пут выше фактической цены акции, то опцион позволяет его владельцу получить прибыль: можно купить акцию по рыночной цене и моментально воспользоваться правом.  
    Опцион-колл — это право купить одну акцию по страйк-цене. 


    \item Алиса и Боб снова подкидывают монетку неограниченное число раз. 
Монетка выпадает стороной $H$ с вероятностью $0.4$ и стороной $T$ с вероятностью $0.6$.
Алиса выигрывает, если последовательность $HHT$ выпадет раньше, а Боб — если раньше выпадет $HTH$.

Рассмотрим множество исходов этого эксперимента с конечным числом букв 
\[
F = \{HHT, HTH, HHHT, THTH, THHT, \dots \}
\]
и производящую функцию этого множества 
\[
s(H, T) = HHT + HTH + HHHT + THTH + THHT + \dots
\]
Здесь аргументы $H$ и $T$ некоммутативны. 
Обозначим $N_H$ — количество выпавших $H$ в эксперименте, $N_T$ — количество выпавших $T$.

\begin{enumerate}
    \item Укажите, как с помощью производных и подстановок раздобыть из функции $s(H, T)$ величины $\P(N_H = 10)$,
    $\P(N_H = 5, N_Y=5)$, $\E(N_H)$, $\E(N_H^3)$, $\E(N_H^2 N_T^3)$.
    \item С помощью метода первого шага составьте систему линейных уравнений, из которой можно найти $s(H, T)$. 
    \item Решите эту систему, предполагая коммутативность $H$ и $T$. 
    \item Завершите вычисление $\P(N_H = 10)$, $\P(N_H = 5, N_T=5)$, $\E(N_H)$, $\E(N_H^3)$, $\E(N_H^2 N_T^3)$.
    \item Убедите проверяющего, что вероятность того, что Боб и Алиса будут бесконечно долго подбрасывать монетку, равна нулю.
\end{enumerate}

Эту задачу можно решать с помощью любого языка программирования с открытым исходным кодом (python, julia, R, \dots).
При этом требуется привести не только и код, и свои рассуждения.
    
\end{enumerate}


Прекрасная неоцениваемая часть в удовольствие:

\begin{enumerate}[resume]

\item Величина $X$ распределена на отрезке $[0;1]$ и на нём имеет функцию плотности $f(t)=3t^2$. 

\begin{enumerate}
 \item Постройте график её функции плотности.
 \item Найдите функцию распределения $X$ и постройте её график.
 \item Найдите функции плотности и функции распределения величины $Y=\ln X$.
\item Найдите $\E(X)$, $\Var(X)$, $\Cov(X, X^2)$. 
\end{enumerate}

 \item Функция плотности случайной величины  $X$ имеет
 вид
 \[
 f(t)=
 \begin{cases}
   \frac{3}{16}t^2,t\in [-2;2] \\
   0, t\notin [-2;2]
 \end{cases}
 \]
 \begin{enumerate}
 \item Постройте график функции плотности.
 \item Найдите $\P(X>1)$, $\E(X)$, $\E(X^2)$, дисперсию $\Var(X)$ и стандартное отклонение $\sigma_X$.
 \item Найдите $\E(X \mid X>0)$, $\E(X^2 \mid X>0)$, $\Var(X \mid X>0)$.
 \item Найдите функции $F^L(x) = \P(X <x)$, $F^R(x) = \P(X \leq x)$ и $u(x) = \P(X=x)$ и нарисуйте их график.
 \item Найдите медиану величины $X$, 40\%-ю квантиль величины $X$.
 \end{enumerate}
 

 \item Прямой убыток от пожара в миллионах рублей равномерно распределен на $[0, 1]$. 
 Если убыток оказывается больше 0.7, то страховая компания выплачивает компенсацию 0.7.
 \begin{enumerate}
 \item Найдите функцию распределения потерь от пожара.
 \item Чему равны ожидаемые потери?
 \end{enumerate}
 
\item В письме своему издателю Фивегу 16 января 1797 года Гёте пишет: «Я намерен предложить господину Фивегу из Берлина 
эпическую поэму «Герман и Доротея» в 2000 гексаметров\ldots{ } С гонораром мы поступим следующим образом:
я передам господину Бёттигеру запечатанный конверт с запрашиваемой мной суммой и буду ожидать 
суммы, предлагаемой господином Фивегом за мой труд. 
Если его предложение окажется ниже запрашиваемой мной суммы, то я отзываю свой конверт нераспечатанным, 
а сделка считается несостоявшейся. 
Если же, напротив, его предложение выше, тогда я не буду запрашивать больше суммы, написанной в моём конверте,
который вскроет господин Бёттигер».


Гёте оценивает поэму в $G$ фридрихсдоров, а Фивег — в $V$ фридрихсдоров. 
Величины $G$ и $V$ независимы и непрерывно распределены на отрезке $[0, 1]$.
Для простоты можно считать, что оба закона распределения равномерны. 

Величину $G$ Гёте передал Бёттигеру в запечатанном конверте. 

\begin{enumerate}
  \item Какую сумму $b$ стоит написать Фивегу в письме Бёттигеру, чтобы максимизировать ожидаемую прибыль?
\end{enumerate}

Рассмотрим альтернативную схему: Гёте явно объявляет Фивегу требуемую сумму $G$ за поэму, 
а затем издатель соглашается или нет.

\begin{enumerate}[resume]
  \item В какой схеме ожидаемый выигрыш издателя выше? 
  \item В какой схеме выше вероятность одобрения сделки?
  \item В чём преимущество оригинальной схемы Гёте?
\end{enumerate}



\end{enumerate}


\end{document}



% проектируемая часть ниже ....