% arara: xelatex
\documentclass[12pt]{article}

% Econometrics, 2024-2025

% \usepackage{physics}

\usepackage{verse}

\usepackage{tikzducks}

\usepackage{tikz} % картинки в tikz
\usepackage{microtype} % свешивание пунктуации

\usepackage{array} % для столбцов фиксированной ширины

\usepackage{indentfirst} % отступ в первом параграфе

\usepackage{sectsty} % для центрирования названий частей
\allsectionsfont{\centering}

\usepackage{amsmath, amsfonts, amssymb} % куча стандартных математических плюшек

\usepackage{comment}

\usepackage[top=2cm, left=1.2cm, right=1.2cm, bottom=2cm]{geometry} % размер текста на странице

\usepackage{lastpage} % чтобы узнать номер последней страницы

\usepackage{enumitem} % дополнительные плюшки для списков
%  например \begin{enumerate}[resume] позволяет продолжить нумерацию в новом списке
\usepackage{caption}

\usepackage{hyperref} % loads url
% \usepackage{url} % to use \url{link to web}
\urlstyle{same}


\newcommand{\smallduck}{\begin{tikzpicture}[scale=0.3]
    \duck[
        cape=black,
        hat=black,
        mask=black
    ]
    \end{tikzpicture}}

\usepackage{fancyhdr} % весёлые колонтитулы
\pagestyle{fancy}
\lhead{}
\chead{}
\rhead{Домашние задания для самураев}
\lfoot{}
\cfoot{}
\rfoot{}

\renewcommand{\headrulewidth}{0.4pt}
\renewcommand{\footrulewidth}{0.4pt}

\usepackage{tcolorbox} % рамочки!

\usepackage{todonotes} % для вставки в документ заметок о том, что осталось сделать
% \todo{Здесь надо коэффициенты исправить}
% \missingfigure{Здесь будет Последний день Помпеи}
% \listoftodos - печатает все поставленные \todo'шки


% более красивые таблицы
\usepackage{booktabs}
% заповеди из докупентации:
% 1. Не используйте вертикальные линни
% 2. Не используйте двойные линии
% 3. Единицы измерения - в шапку таблицы
% 4. Не сокращайте .1 вместо 0.1
% 5. Повторяющееся значение повторяйте, а не говорите "то же"


\setcounter{MaxMatrixCols}{20}
% by crazy default pmatrix supports only 10 cols :)


\usepackage{fontspec}
\usepackage{libertine}
\usepackage{polyglossia}

\setmainlanguage{russian}
\setotherlanguages{english}

% download "Linux Libertine" fonts:
% http://www.linuxlibertine.org/index.php?id=91&L=1
% \setmainfont{Linux Libertine O} % or Helvetica, Arial, Cambria
% why do we need \newfontfamily:
% http://tex.stackexchange.com/questions/91507/
% \newfontfamily{\cyrillicfonttt}{Linux Libertine O}

\AddEnumerateCounter{\asbuk}{\russian@alph}{щ} % для списков с русскими буквами
% \setlist[enumerate, 2]{label=\asbuk*),ref=\asbuk*}

%% эконометрические сокращения
\DeclareMathOperator{\Cov}{\mathbb{C}ov}
\DeclareMathOperator{\Corr}{\mathbb{C}orr}
\DeclareMathOperator{\Var}{\mathbb{V}ar}
\DeclareMathOperator{\hVar}{\widehat{\mathbb{V}ar}}
\DeclareMathOperator{\col}{col}
\DeclareMathOperator{\row}{row}

\let\P\relax
\DeclareMathOperator{\P}{\mathbb{P}}

\DeclareMathOperator{\E}{\mathbb{E}}
% \DeclareMathOperator{\tr}{trace}
\DeclareMathOperator{\card}{card}

\DeclareMathOperator{\Convex}{Convex}

\newcommand \cN{\mathcal{N}}
\newcommand \RR{\mathbb{R}}
\newcommand \NN{\mathbb{N}}

\newcommand{\dU}{\mathrm{Unif}}
\newcommand{\dUnif}{\mathrm{Unif}}

\DeclareMathOperator{\loss}{loss}

\newcommand{\hy}{\hat y}
\newcommand{\hb}{\hat\beta}

\usepackage{mathtools}
\DeclarePairedDelimiter{\norm}{\lVert}{\rVert}
\DeclarePairedDelimiter{\abs}{\lvert}{\rvert}
\DeclarePairedDelimiter{\scalp}{\langle}{\rangle}


\begin{document}

\section*{Домашнее задание 1}

Дедлайн: 2025-03-20, 21:00.

Оцениваемая часть:

\begin{enumerate}
\item В красном мешке у Деда Мороза 5 красных и 4 синих шара, а в синем мешке — 3 красных и 10 синих шаров. 
Сначала Дед Мороз выбирает один из мешков равновероятно. 
Затем Дед Мороз достаёт из выбранного мешка один шар.
А затем Дед Мороз достаёт ещё два шара из \emph{другого} мешка.

Обозначим $R$ — общее число красных извлечённых шаров, и $B$ — общее число синих шаров.

\begin{enumerate}
    \item Составьте табличку распределения случайной величины $R$.
    \item Найдите $\P(R \text{ — чётное})$, $\E(R)$, $\E(2B + 7)$, $\E(R\cdot B)$.
    \item Найдите $\P(R \geq 1, B \geq 1)$, $\E(R \cdot I(B\geq 1))$.
\end{enumerate}

Напоминалочка: $I(A)$ — индикатор события $A$, случайная величина, равная $1$, если событие $A$ произошло и $0$ — иначе.


\item У Илона Маска две монетки: $A$-монетка выпадает орлом с вероятностью $0.3$, $B$-монетка выпадает орлом с вероятностью $0.4$.
Каждая из монеток выпадает либо решкой, либо орлом. 
Всего Илон делает $100$ подбрасываний. 
Сначала Илон Маск подбрасывает монетку $A$. 
Затем он действует по простому правилу: если выпал орёл, то следующей будет подброшена монетка $A$, если выпала решка,
то следующией будет подброшена монетка $B$.
Обозначим $X$ — общее число выпавших орлов, $Y$ — общее число орлов выпавших на монетке $B$.

\begin{enumerate}
    \item Найдите $\E(X)$ и $\E(Y)$.
    \item Найдите $\E(XY)$.
\end{enumerate}

\end{enumerate}

Прекрасная неоцениваемая часть в удовольствие:

\begin{enumerate}[resume]
    \item У Маши две монетки: золотая и серебряная.
Сначала Маша подкидывает золотую монетку.
Если золотая монетка выпала орлом, то Маша подкидывает серебряную монетку один раз.
Если золотая монетка выпала решкой — то подкидывает серебряную два раза.

Пусть $X$ — общее количество выпавших орлов на золотой и серебряной монетках.

\begin{enumerate}
\item Найдите все возможные значения $X$ и их вероятности.
\item Каково ожидаемое количество выпавших орлов?
\end{enumerate}

    \item Вспомним свойство аддитивности вероятности. 
    $A$: Если задан набор несовместных событий $A_1$, $A_2$, \dots, ($A_i \cap A_j = \emptyset$ при $i\neq j$),
    то $\P(\cup A_i) = \sum_i \P(A_i)$.

    Докажите, что свойство аддитивности эквивалентно свойству $B$ и свойству $C$.

    $B$: Если задан набор вложенных событий $B_1 \subseteq B_2 \subseteq B_3 \dots$, то $\lim_i \P(B_i) = \P(\lim_i B_i)$.

    $C$: Если задан набор вложенных событий $\dots C_3 \subseteq C_2 \subseteq C_1$, то $\lim_i \P(C_i) = \P(\lim_i C_i)$.

    \item В шкатулке у Маши 100 пар серёжек. 
    Каждый день утром она выбирает одну пару наугад, носит ее, а вечером возвращает в шкатулку. 
    Проходит год.
    \begin{enumerate}
    \item Сколько в среднем пар окажутся ни разу не надетыми?
    \item Сколько в среднем пар окажутся надетыми не менее двух раз?
\end{enumerate}
    \item  Над озером взлетело 20 уток. 
    Каждый из 10 охотников один раз стреляет в случайно выбираемую им утку. 
    Охотники целятся одновременно, поэтому несколько охотников могут выбрать одну и ту же утку.
    Величина $Y$ — количество выживших уток, $X$ — количество попавших в цель охотников. 
   
   \begin{enumerate}
     \item Найдите $\E(X)$, $\E(Y)$, если охотники стреляют без промаха. 
     \item Как изменятся ответы, если вероятность попадания равна $0.7$?
   \end{enumerate}
   
\end{enumerate}
    

\section*{Домашнее задание 2}

Дедлайн: 2025-03-27, 21:00.

Оцениваемая часть:

\begin{enumerate}
\item 
\end{enumerate}

Прекрасная неоцениваемая часть в удовольствие:

\begin{enumerate}
    \item 
\end{enumerate}

\end{document}



% проектируемая часть ниже ....