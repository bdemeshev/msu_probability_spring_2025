% arara: xelatex
\documentclass[12pt]{article}

% \usepackage{physics}

\usepackage{hyperref}
\hypersetup{
    colorlinks=true,
    linkcolor=blue,
    filecolor=magenta,      
    urlcolor=cyan,
    pdftitle={Overleaf Example},
    pdfpagemode=FullScreen,
    }

\usepackage{tikzducks}

\usepackage{tikz} % картинки в tikz
\usepackage{microtype} % свешивание пунктуации

\usepackage{array} % для столбцов фиксированной ширины

\usepackage{indentfirst} % отступ в первом параграфе

\usepackage{sectsty} % для центрирования названий частей
\allsectionsfont{\centering}

\usepackage{amsmath, amsfonts, amssymb} % куча стандартных математических плюшек

\usepackage{mathtools}
\usepackage{comment}

\usepackage[top=2cm, left=1.2cm, right=1.2cm, bottom=2cm]{geometry} % размер текста на странице

\usepackage{lastpage} % чтобы узнать номер последней страницы

\usepackage{enumitem} % дополнительные плюшки для списков
%  например \begin{enumerate}[resume] позволяет продолжить нумерацию в новом списке
\usepackage{caption}

\usepackage{url} % to use \url{link to web}


\newcommand{\smallduck}{\begin{tikzpicture}[scale=0.3]
    \duck[
        cape=black,
        hat=black,
        mask=black
    ]
    \end{tikzpicture}}

\usepackage{fancyhdr} % весёлые колонтитулы
\pagestyle{fancy}
\lhead{}
\chead{}
\rhead{Демо версия, май 2025}
\lfoot{}
\cfoot{}
\rfoot{}

\renewcommand{\headrulewidth}{0.4pt}
\renewcommand{\footrulewidth}{0.4pt}

\usepackage{tcolorbox} % рамочки!

\usepackage{todonotes} % для вставки в документ заметок о том, что осталось сделать
% \todo{Здесь надо коэффициенты исправить}
% \missingfigure{Здесь будет Последний день Помпеи}
% \listoftodos - печатает все поставленные \todo'шки


% более красивые таблицы
\usepackage{booktabs}
% заповеди из докупентации:
% 1. Не используйте вертикальные линни
% 2. Не используйте двойные линии
% 3. Единицы измерения - в шапку таблицы
% 4. Не сокращайте .1 вместо 0.1
% 5. Повторяющееся значение повторяйте, а не говорите "то же"


\setcounter{MaxMatrixCols}{20}
% by crazy default pmatrix supports only 10 cols :)


\usepackage{fontspec}
\usepackage{libertine}
\usepackage{polyglossia}

\setmainlanguage{russian}
\setotherlanguages{english}

% download "Linux Libertine" fonts:
% http://www.linuxlibertine.org/index.php?id=91&L=1
% \setmainfont{Linux Libertine O} % or Helvetica, Arial, Cambria
% why do we need \newfontfamily:
% http://tex.stackexchange.com/questions/91507/
% \newfontfamily{\cyrillicfonttt}{Linux Libertine O}

\AddEnumerateCounter{\asbuk}{\russian@alph}{щ} % для списков с русскими буквами
\setlist[enumerate, 2]{label=\asbuk*),ref=\asbuk*}

%% эконометрические сокращения
\DeclareMathOperator{\Cov}{\mathbb{C}ov}
\DeclareMathOperator{\BestLin}{BestLin}
\DeclareMathOperator{\Corr}{\mathbb{C}orr}
\DeclareMathOperator{\Var}{\mathbb{V}ar}
\DeclareMathOperator{\pCorr}{\mathrm{pCorr}}
\DeclareMathOperator{\col}{col}
\DeclareMathOperator{\row}{row}

\let\P\relax
\DeclareMathOperator{\P}{\mathbb{P}}

\DeclarePairedDelimiter{\abs}{\lvert}{\rvert}

\let\H\relax
\DeclareMathOperator{\H}{\mathbb{H}}


\DeclareMathOperator{\E}{\mathbb{E}}
% \DeclareMathOperator{\tr}{trace}
\DeclareMathOperator{\card}{card}

\DeclareMathOperator{\Convex}{Convex}

\newcommand \cN{\mathcal{N}}
\newcommand \dN{\mathcal{N}}
\newcommand \dBin{\mathrm{Bin}}


\newcommand \RR{\mathbb{R}}
\newcommand \NN{\mathbb{N}}

\newcommand{\dBern}{\mathrm{Bern}}



\begin{document}

\section*{Формат}

В контрольной работе будет 6 задач. 
Задачи имеют равный вес. 
Продолжительность работы 120 минут. 
Можно использовать чит-лист А4 и простой калькулятор.

\section*{Демо «Хонсю»}
\begin{enumerate}

    \item % двумерная плотность
    Случайный вектор $(X, Y)$ имеет функцию плотности 
    \[
    f(x, y) = \begin{cases}
        cx + y, \text{ если } x\in[0,1], y \in [0, 1] \\
        0, \text{ иначе}.    
    \end{cases}
    \]
    \begin{enumerate}
        \item Найдите константу $c$.
        \item Найдите функцию плотности $f_X(x)$ и условную функцию плотности $f(y\mid x)$.
        \item Найдите $\E(X^3)$, $\Corr(X, Y)$ и $\E(Y \mid X)$.
    \end{enumerate}
    
    

    \item % энтропия
Величина $X$ имеет экспоненциальное распределение с интенсивностью $\lambda = 1$.
\begin{enumerate}
    \item Найдите энтропию величины $X$.
    \item Сколько в среднем вопросов нужно задать, чтобы угадать $X$ с точностью до $10^{-6}$, при использовании оптимальной стратегии?
    \item А сколько в среднем вопросов придётся задать, чтобы угадать $X$ с точностью до $10^{-6}$, если ошибочно верить, что она распределена экспоненциально с интенсивностью $\lambda = 2$?
\end{enumerate}


    \item % одномерное нормальное
    Случайная величина $X$ имеет нормальное распределение $\cN(3, 10)$. 
    Обозначим её функции распределения и плотности как $F$ и $f$, соответственно. 
    \begin{enumerate}
        \item Найдите $F(5)$ и $f(0)$.
        \item Найдите точку экстремума $f$ и точки перегиба $f$.
        \item Схематично постройте графики $F$ и $f$ на соседних графиках друг над другом. 
        \item Найдите $\alpha$ и $\beta$, если известно, что $Y = (X - \alpha)/\beta \sim \cN(0, 1)$.  
    \end{enumerate}
    
    \item % многомерное нормальное и ков. матрица
    Вектор $X = (X_1, X_2, X_3)$ имеет многомерное нормальное распределение $\cN(\mu, C)$,
    где $\mu = (1, 2, 3)$ и $C = \begin{pmatrix}
        10 & -1 & 0 \\
         & 20 & 1 \\
         & & 30 \\
    \end{pmatrix}$. 

    Рассмотрим вектор $Y = (Y_1, Y_2) = (X_1 + X_2 + X_3, 2X_2 - X_3)$.
    \begin{enumerate}
        \item Найдите ожидание $\E Y$ и ковариационную матрицу $\Var Y$. Как распределён вектор $Y$?
        \item Найдите $\E(X_1 X_2 X_3)$ и $\Cov(X_1^2, X_3)$.
    \end{enumerate}
    
    \item % пуассоновский поток
    Такси прибывают на остановку пуассоновским потоком с интенсивностью $\lambda = 10$ в час. 
    Пусть $Y_t$ — количество такси, прибывших от начала наблюдения до момента времени $t$.
    \begin{enumerate}
        \item Найдите функцию $\E(Y_{5} - Y_t)$ и постройте её график.
        \item Найдите функцию $\Var(Y_{5} - Y_t)$ и постройте её график.
        \item Для вектора $X = (Y_1, Y_5, Y_{10})$ найдите $\E X$ и $\Var X$.
    \end{enumerate}

    \item % разное
Случайная величина $X$ имеет функцию распределения
\[
F(x) = \begin{cases}
    0, \text{ если } x< 0, \\
    x/4, \text{ если } x \in [0, 1) \\
    x/4 + 1/4, \text{ если } x\in [1, 3), \\
    b, \text{ если } x \geq 3. 
\end{cases}
\]
\begin{enumerate}
    \item Найдите $b$. 
    \item Найдите $\P(X = 1)$ и $\P(X = 2)$.
    \item Найдите $\E(X)$, $\Var(X)$ и $\Corr(X, X^2)$.
\end{enumerate}

\end{enumerate}


\section*{Демо «Сикоку»}

\begin{enumerate}
    \item % двумерная плотность
    Случайный вектор $(X, Y)$ имеет функцию плотности 
    \[
    f(x, y) = \begin{cases}
        4xy, \text{ если } x\in[0,1], y \in [0, 1] \\
        0, \text{ иначе}.    
    \end{cases}
    \]
    \begin{enumerate}
        \item Найдите $\P(X + Y < 1)$, $\P(X + 2Y < 1 \mid X + Y < 1)$.
        \item Найдите функцию плотности $f_X(x)$ и условную функцию плотности $f(y\mid x)$.
        \item Найдите совместную функцию плотности $f_{UV}(u, v)$, где $U = 2X + 3Y$, $V = 2X - 4Y$.
    Зависимы ли величины $U$ и $V$?
    \end{enumerate}

    

\item % энтропия
Величина $X$ имеет нормальное распределение $\cN(1, 4)$.
\begin{enumerate}
    \item Найдите энтропию величины $X$.
    \item Сколько в среднем вопросов нужно задать, чтобы угадать $X$ с точностью до $10^{-6}$, при использовании оптимальной стратегии?
    \item А сколько в среднем вопросов придётся задать, чтобы угадать $X$ с точностью до $10^{-6}$, если ошибочно верить, что она распределена $\cN(2, 4)$?
\end{enumerate}



\item % одномерное нормальное
Случайная величина $X$ имеет функцию плотности $f(x) = c \cdot \exp(4x - x^2/32)$,
где $c$ — некоторая константа. 
\begin{enumerate}
    \item Как распределена случайная величина $X$?
    \item Найдите константу $c$.
    \item Найдите $\E(X^4)$, $\E(X^3)$, $\Cov(X^3, X)$.
\end{enumerate}

Подсказка: если представить $X$ как $X = \mu + Y$, то $\E(Y) = 0$ и можно будет применить лемму Стейна $\E(Y g(Y)) = \sigma^2 \E(g'(Y))$.

\item % многомерное нормальное и ков. матрица
Вектор $X$ имеет многомерное нормальное распределение $\cN(\mu, C)$ с функцией плотности $f(x)$.
Рассмотрим функцию $h(x) = \ln f(x)$, где $x\in \RR^n$.
\begin{enumerate}
    \item В какой точке функция $h(x)$ достигает своего максимума?
    \item Чему равна матрица Гессе функции $h(x)$?
\end{enumerate}
У случайного вектора $Y$ функция плотности равна $f(y_1, y_2) = c \cdot \exp(-4y_1^2 - 6y_2^2 + 2y_1 + 20y_2)$.
\begin{enumerate}[resume]
    \item Найдите $\E Y$, $\Var Y$. Как распределён вектор $Y$?
    \item Найдите $\Corr(Y_1, Y_2)$.
\end{enumerate}

\item % пуассоновский поток
Такси прибывают на остановку пуассоновским потоком с интенсивностью $\lambda = 10$ в час. 
Пусть $Y_t$ — количество такси, прибывших от начала наблюдения до момента времени $t$.
\begin{enumerate}
    \item Найдите $\P(Y_{0.1} = 2)$, $\P(Y_{0.2} = 2 \mid Y_{0.1} = 1)$.
    \item Найдите $\Corr(Y_1, Y_7)$.
    \item Я только что пришёл на остановку. Какова вероятность того, что следующее такси я увижу раньше, чем за 5 минут?
\end{enumerate}

\item % разное
Илон Маск подбрасывает правильную монетку два раза. 
Рассмотрим три индикатора: $I_1$ — индикатор того, что в первом броске выпал орёл,
$I_2$ — индикатор того, что во втором броске выпал орёл, 
$I_3$ — индикатор того, что результаты двух бросков одинаковые.
\begin{enumerate}
    \item Найдите $\BestLin(I_3 \mid I_1)$.
    \item Найдите $\BestLin(I_3 \mid I_1, I_2)$.
    \item Найдите $\E(I_3 \mid I_1, I_2)$.
\end{enumerate}

Уточнение: конечно, функция $\BestLin(Y \mid X, Z)$ обязана иметь вид $\alpha + \beta X + \gamma Z$.

\end{enumerate}


\end{document}

\section*{«Тыква» решение}

\begin{enumerate}
    \item 
    \begin{enumerate}
        \item $\P(\text{Маша спрогнозирует ясно}) = 0.3\cdot 0.8 + 0.7\cdot 0.2 = 0.38$.
        \item $\P(\text{прогнозы совпадут}) = 0.9$.
        \item $\P(\text{ясно} \mid \text{Маша спрогнозировала ясно}) = 0.3\cdot 0.8 / 0.38 \approx 0.63$.
        \item $\P(\text{ясно} \mid \text{Вовочка спрогнозировал ясно}) = 111/202$.
    \end{enumerate}
    \item     
    \item 
        Пусть величина $X_i$ — индикатор события «$i$-я и $(i+1)$-я монетки выпали решкой». 
        Тогда $X_i \sim \dBern(1/4)$, $\E(X_i) = 1/4$, $\Var(X_i) = 3/16$. 
        Заметим, что $X = X_1 + X_2 + \ldots + X_{29}$ — количество пар решек подряд. 
        Для количества выплат получаем $\E(X) = 29/4$, $\Var(X) = 29\cdot 3/16 + ... $.
        
        Для суммы выплат $Y = 100X$, $\E(Y) = 725$, $\Var(Y) = ...$.
    \item 
    \begin{enumerate}
        \item $\E(X) = \int_0^2 x^2/2 \, dx = 4/3$, $\P(X < 1) = \int_0^1 x/2 \, dx = 1/4$, $\E(X \mid X < 1) = \int_0^1 x^2/2 \, dx / \P(X < 1) = 2/3$.
        \item $m(t) = \E(e^{tX}) = \int_0^2 e^{tx} x/2 \, dx = \begin{cases}
            (2te^{2t} - e^{2t} + 1)/2t^2, \text{ если } t \neq 0 \\
            1, \text{ если } t = 0
        \end{cases}$.
        \item $F_Y(y) = \P(Y < y) = \P(X < y/2) =  \begin{cases}
            0, \text{ если } y < 0, \\
            y^2/16, \text{ если } y \in [0;4], \\
            1, \text{ если } y > 4. \\
        \end{cases}$.
    \end{enumerate}

\end{enumerate}

\section*{Демо «Летучая мышь»}
\begin{enumerate}
    \item % комбинаторика
    В колоде 53 карты: один джокер, которого можно засчитать за любую карту, и 13 достоинств от двойки до туза по 4 масти. 
    Игрок случайным образом получает 5 карт из колоды.

    \begin{enumerate}
        \item Какова вероятность того, что полученную комбинацию можно интепретировать как фулл-хаус (три карты разных мастей одного достоинства и ещё две карты разных мастей другого достоинства)?
        \item Какова вероятность того, что полученную комбинацию можно интепретировать как стрит-флэш (пять идущих подряд карт одной масти)?
    \end{enumerate}
    
    \item % условная вероятность
   На побережье одна за одной набегают волны. 
   Высота каждой волны — равномерная на $[0; 1]$ случайная величина. 
   Высоты волн независимы. 
   Пираты называют волну «большой», если она больше предыдущей и больше следующей. 
   Пираты называют волну «рекордной», если она больше всех предыдущих волн от начала наблюдения. 
   Обозначим события $B_i = \{i-\text{я волна была большой}\}$ и $R_i = \{i-\text{я волна была рекордной}\}$.
   
   \begin{enumerate}
    \item Найдите $\P(B_1 \mid B_2)$, $\P(B_1 \mid B_3)$.
    \item Найдите $\P(R_{2024} \mid R_{2025})$, $\P(R_{2024} \mid B_{2024})$.
    \item Укажите любую функцию $a(n)$ такую, что $a(n) = O(\E(X_n))$, где $X_n$ — количество рекордных волн среди $n$ волн. 
   \end{enumerate}

    \item % метод первого шага
    Глеб Жеглов каждый день ловит одного преступника. 
    Однако с вероятностью $0.05$ вместо одного пойманного на преступный путь встают $w$ новых граждан. 
    Изначально в городе живёт $n$ преступников.
    Сколько дней в среднем пройдёт до полного искоренения преступности в городе?
    \begin{enumerate}
        \item Решите задачу при $n = 1$ и $w = 1$.
        \item Решите задачу при произвольных $n$ и $w$.
    \end{enumerate}
    \item % геометрическая вероятность и плотность 
    На единичной окружности с центром в начале координат (не внутри!) в случайные точки приползли три муравья.
    Три точки  независимы и равномерно распределены по окружности. 
    Два муравья могут общаться друг с другом, если угол между ними меньше прямого. 

\begin{enumerate}
    \item Какова вероятность того, что все три муравья смогут не перемещаясь общаться друг с другом (возможно
    через посредника)?
    \item Какова вероятность того, что все три муравья смогут не перемещаясь общаться друг с другом через посредника, 
    если угол между муравьём один и муравьём два больше прямого?
    \item Найдите функцию плотности координат первого муравья. 
\end{enumerate}
    \item % энтропия, дисперсии и ковариации 
    Величины $X_n$ независимы и равны $(+1)$ с вероятностью $1/2n$, $(-1)$ — с вероятностью $1/2n$, $0$ — с вероятностью $1-1/n$.
    Определим $Y_n = \sum_{i=1}^n \sqrt{i} X_i / n$.

    Оцените вероятность $\P(\abs{Y_n} \geq 1)$ с помощью неравенства Чебышёва.

    \item % совместная плотность 
    Пара величин $(X, Y)$ имеет совместную функцию плотности $f(x, y) = x + by$ на квадрате $[0;1]\times[0;1]$.
    \begin{enumerate}
        \item Найдите значение параметра $b$. 
        \item Найдите функцию плотности величины $X$ и энтропию $\H(X)$.
        \item Найдите корреляцию $\Corr(X, Y)$.
    \end{enumerate}

\end{enumerate}

\end{document}

%Из определения условного ожидания, $\E(X \mid A) = \E(X \cdot I_A) / \P(A)$, 
%легко получаются определения условной дисперсии, $\Var(X \mid A) = \E(X^2 \mid A) - (\E(X \mid A))^2$,
%условной ковариации $\Cov(X, Y \mid A) = \E(XY \mid A) - \E(X \mid A) \E(Y \mid A)$ и даже корреляции, 
%$\Corr(X, Y \mid A) = \Cov(X, Y \mid A) / \sqrt{\Var(X \mid A)\Var(Y \mid A)}$.

% здесь проектируемая часть



\end{document}

