% arara: xelatex
\documentclass[12pt]{article}

% \usepackage{physics}

\usepackage{hyperref}
\hypersetup{
    colorlinks=true,
    linkcolor=blue,
    filecolor=magenta,      
    urlcolor=cyan,
    pdftitle={Overleaf Example},
    pdfpagemode=FullScreen,
    }

\usepackage{tikzducks}

\usepackage{tikz} % картинки в tikz
\usepackage{microtype} % свешивание пунктуации

\usepackage{array} % для столбцов фиксированной ширины

\usepackage{indentfirst} % отступ в первом параграфе

\usepackage{sectsty} % для центрирования названий частей
\allsectionsfont{\centering}

\usepackage{amsmath, amsfonts, amssymb} % куча стандартных математических плюшек

\usepackage{mathtools}
\usepackage{comment}

\usepackage[top=2cm, left=1.2cm, right=1.2cm, bottom=2cm]{geometry} % размер текста на странице

\usepackage{lastpage} % чтобы узнать номер последней страницы

\usepackage{enumitem} % дополнительные плюшки для списков
%  например \begin{enumerate}[resume] позволяет продолжить нумерацию в новом списке
\usepackage{caption}

\usepackage{url} % to use \url{link to web}


\newcommand{\smallduck}{\begin{tikzpicture}[scale=0.3]
    \duck[
        cape=black,
        hat=black,
        mask=black
    ]
    \end{tikzpicture}}

\usepackage{fancyhdr} % весёлые колонтитулы
\pagestyle{fancy}
\lhead{}
\chead{}
\rhead{Запоздалая контроша :)}
\lfoot{}
\cfoot{}
\rfoot{}

\renewcommand{\headrulewidth}{0.4pt}
\renewcommand{\footrulewidth}{0.4pt}

\usepackage{tcolorbox} % рамочки!

\usepackage{todonotes} % для вставки в документ заметок о том, что осталось сделать
% \todo{Здесь надо коэффициенты исправить}
% \missingfigure{Здесь будет Последний день Помпеи}
% \listoftodos - печатает все поставленные \todo'шки


% более красивые таблицы
\usepackage{booktabs}
% заповеди из докупентации:
% 1. Не используйте вертикальные линни
% 2. Не используйте двойные линии
% 3. Единицы измерения - в шапку таблицы
% 4. Не сокращайте .1 вместо 0.1
% 5. Повторяющееся значение повторяйте, а не говорите "то же"


\setcounter{MaxMatrixCols}{20}
% by crazy default pmatrix supports only 10 cols :)


\usepackage{fontspec}
\usepackage{libertine}
\usepackage{polyglossia}

\setmainlanguage{russian}
\setotherlanguages{english}

% download "Linux Libertine" fonts:
% http://www.linuxlibertine.org/index.php?id=91&L=1
% \setmainfont{Linux Libertine O} % or Helvetica, Arial, Cambria
% why do we need \newfontfamily:
% http://tex.stackexchange.com/questions/91507/
% \newfontfamily{\cyrillicfonttt}{Linux Libertine O}

\AddEnumerateCounter{\asbuk}{\russian@alph}{щ} % для списков с русскими буквами
\setlist[enumerate, 2]{label=\asbuk*),ref=\asbuk*}

%% эконометрические сокращения
\DeclareMathOperator{\Cov}{\mathbb{C}ov}
\DeclareMathOperator{\BestLin}{BestLin}
\DeclareMathOperator{\Corr}{\mathbb{C}orr}
\DeclareMathOperator{\Var}{\mathbb{V}ar}
\DeclareMathOperator{\pCorr}{\mathrm{pCorr}}
\DeclareMathOperator{\col}{col}
\DeclareMathOperator{\row}{row}

\let\P\relax
\DeclareMathOperator{\P}{\mathbb{P}}

\DeclarePairedDelimiter{\abs}{\lvert}{\rvert}

\let\H\relax
\DeclareMathOperator{\H}{\mathbb{H}}


\DeclareMathOperator{\E}{\mathbb{E}}
% \DeclareMathOperator{\tr}{trace}
\DeclareMathOperator{\card}{card}

\DeclareMathOperator{\Convex}{Convex}

\newcommand \cN{\mathcal{N}}
\newcommand \dN{\mathcal{N}}
\newcommand \dBin{\mathrm{Bin}}


\newcommand \RR{\mathbb{R}}
\newcommand \NN{\mathbb{N}}

\newcommand{\dBern}{\mathrm{Bern}}



\begin{document}

Можно использовать чит-лист А4 и простой калькулятор, продолжительность: 120 минут.

\begin{enumerate}

    \item % двумерная плотность
    Случайный вектор $(X, Y)$ имеет функцию плотности 
    \[
    f(x, y) = \begin{cases}
        2x^2 + y^2, \text{ если } x\in[0, 1], y \in [0, 1] \\
        0, \text{ иначе}.    
    \end{cases}
    \]
    \begin{enumerate}
        \item {[3]} Найдите функцию плотности $f_X(x)$ и условную функцию плотности $f(y \mid x)$.
        \item {[5]} Найдите $\Corr(X, Y)$ и $\E(Y \mid X)$.
        \item {[2]} Найдите значение функции распределения $F(0.5, 0.5)$.
    \end{enumerate}
    
    

    \item % энтропия
Величина $X$ имеет функцию плотности $f(x) = 2x$ на отрезке $[0, 1]$.
\begin{enumerate}
    \item {[4]} Найдите энтропию величины $X$.
    \item {[4]} Сколько в среднем вопросов нужно задать, чтобы угадать $X$ с точностью до $10^{-6}$, при использовании оптимальной стратегии?
    \item {[2]} Как примерно выглядит первый вопрос для стратегии из пункта (б)?
\end{enumerate}


    \item % одномерное нормальное
    Случайная величина $X$ имеет нормальное распределение $\cN(3, 16)$. 
    \begin{enumerate}
        \item {[5]} Найдите $\E(X^2)$, $\Cov(X, X^2)$.
        \item {[5]} Найдите вероятности $\P(X > 3)$, $\P(X > 7)$.
    \end{enumerate}

    Уточнение: выразите все вероятности из пункта (б) с помощью стандартной нормальной функции распределения и найдите их, используя таблицы.
    
    \item % многомерное нормальное и ков. матрица
    Вектор $X = (X_1, X_2, X_3)$ имеет многомерное нормальное распределение $\cN(\mu, C)$,
    где $\mu = (1, 2, 3)$ и $C = \begin{pmatrix}
        40 & 1 & -1 \\
         & 20 & 0 \\
         & & 30 \\
    \end{pmatrix}$. 

    Рассмотрим вектор $Y = (Y_1, Y_2) = (X_1 - X_2, X_2 - X_3)$.
    \begin{enumerate}
        \item {[4]} Найдите ожидание $\E Y$ и ковариационную матрицу $\Var Y$. Как распределён вектор $Y$?
        \item {[6]} Найдите $\E(X_1 X_2 X_3)$ и $\Cov(X_1^2, X_3)$.
    \end{enumerate}
    
    \item % пуассоновский поток
    Мимо комиссара Жибера независимыми пуассоновскими потоками проезжают такси белого и жёлтого цвета. 

   Белые — с интенсивностью $\lambda_W = 2$ за 10 минут, жёлтые — с интенсивностью $\lambda_Y = 3$ за 10 минут.
    \begin{enumerate}
        \item {[2]} Сколько в среднем ждать первое приехавшее такси (любого цвета)?
        \item {[4]} Какова вероятность того, что первое жёлтое такси приедет раньше первого белого?
        \item {[4]} Какова вероятность того, что до первого белого приедет ровно два жёлтых такси?
    \end{enumerate}

    \item % разное
    Оставшееся количество правильного мёда $X$ имеет функцию плотности $f(x) = 2x$ на отрезке $[0, 1]$ (кг).
    У Винни-Пуха в голове опилки и он не может запомнить числа больше, чем $0.5$.
    Если $X > 0.5$, то вместо точного значения $X$ Винни-Пух запоминает равновероятно либо $0.1$, либо $0.2$. 
    Обозначим с помощью $Y$ значение, которое запомнил Винни-Пух. 

    \begin{enumerate}
        \item {[4]} Найдите функцию распределения $Y$.
        \item {[6]} Найдите $\E(Y)$ и $\Var(Y)$.
    \end{enumerate}

\end{enumerate}


\end{document}

